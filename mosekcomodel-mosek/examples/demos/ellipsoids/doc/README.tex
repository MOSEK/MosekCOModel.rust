\documentclass{article}

\RequirePackage{mathtools}
\RequirePackage{amsfonts}

\begin{document}




\section{Maximizing and minimizing ellipsoid volume}

An ellipse given by
\begin{eqnarray}
    E_{A,b} &=& \{ x | \  || Ax+b ||^2 \leq 1\},\  A\in\mathcal{S}^n_+
\end{eqnarray}
has a volume that is proportional to
\[
    \det(A)
\]
This is not convex in $A$, but $\det(A)^{1/n}$ is, and the inequality
\begin{eqnarray}
    t\leq \det(A)^{1/n}
\end{eqnarray}
can be modelled by a semidefinite cone and a primal geometric mean cone.
\begin{eqnarray}
    t &\leq& 
    \left[
        \begin{array}{cc}
            X & Z \\
            Z^T & \mathrm{Diag}(Z)
        \end{array}
    \right]
    \in\mathcal{S}^{2n}_+ \\
    Z & \in & \mathrm{Tril}^n \\
    t & \leq & \left( \prod^n_{i=1} Z_{ii}\right)^{1/n}
\end{eqnarray}
See Nemirovski and Ben-Tal, 4.2.1. Lemma on Schur complement. 

Maximizing $t$ will now maximize the volume. Conversely, maximizing $t$ in 
\begin{eqnarray}
    t &\leq& \det(A^{-1})^{1/n}
\end{eqnarray}
will minimize the volume.

Note that if we use the squared representation $A^2$, then
\begin{eqnarray}
    t &\leq& \det(A^{2})^{1/n} = \det(A)^{2/n}
\end{eqnarray}
to maximizing $t$ will still maximize the volume of $A$.


\section{Contained and containing ellipses}

See Nemirovski and Ben-Tal.

\subsection{Find ellipsoid containing ellipsoid}
Given an ellipse 
\[
  C = \{ x\in\mathbb{R}^n : ||A(x-b)||^2 \leq 1 \}
\]
and an ellipse
\[
  E = \{ Zx+w : ||x||_2 \leq 1 \}
\]
We will assume that $A,Z \in\mathcal{S}^n_+$ (symmetric, positive semidefinite matrix of dimension $n$).
Now, 
\begin{eqnarray}
    E \subset C 
    & \Leftrightarrow & ||x||_2^2 \leq 1 \Rightarrow || A((Zx+w)-b) ||_2^2 \leq 1 \\
    & \Leftrightarrow & ||x||_2^2 < t^2 \Rightarrow ||A(Zx+t(w-b))||_2^2 \leq t^2 
%   & \Leftrightarrow & ||x||_2^2 < t^2 \Rightarrow (Zx+t(w-b))^TA^TA(Zx+t(w-b)) \leq t^2 
\end{eqnarray}
By the $\mathcal{S}$-lemma:
\begin{eqnarray}
    & \Leftrightarrow & \exists\lambda\geq 0: \left( t^2 - || (Zx+t(w-b))^TA ||_2^2 -\lambda(t^2 - ||x||_2^2) \right) \geq 0 \\
    &\Leftrightarrow & 
    \left[
        \begin{array}{cc}
            1-\lambda-|| A(w-b) ||^2 & -(w-b)A^TAZ) \\
            -Z^TY^TY(w-b)            & \lambda I_n - Z^TA^TAZ
        \end{array}
    \right] \in\mathcal{S}^{n+1}_+\\
    &\Leftrightarrow & 
    \left[
        \begin{array}{cc}
            1-\lambda & . \\
            . & \lambda I_n 
        \end{array}
    \right]
    -
    \left[
        \begin{array}{c}
            (A(w-b))^T \\
            (AZ)^T
        \end{array}
    \right]
    \left[
        \begin{array}{cc}
            A(w-b) & AZ
        \end{array}
    \right]\in\mathcal{S}^{n+1}_+
\end{eqnarray}
By lemma on Schurs complement
\begin{eqnarray}
& \Leftrightarrow & 
    P(\lambda) := \left[
        \begin{array}{ccc}
            I_n & A(w-b) & AZ \\
            (A(w-b))^T & 1-\lambda & . \\
            (AZ)^T & . & \lambda I_n
        \end{array}
    \right]\in\mathcal{S}^{2n+1}_+
\end{eqnarray}
Furthermore, since $A\in\mathcal{S}^n_+$
\begin{eqnarray}
S := \left[
    \begin{array}{ccc}
     A^{-1} \\
     & 1 \\
     && I_n
    \end{array}
\right]\in\mathcal{S}^{2n+1}_+
\end{eqnarray}
we have 
\begin{eqnarray}
    P(\lambda) \in \mathcal{S}^{2n+1}_+ 
    & \Leftrightarrow &
    S^T P(\lambda) S \in\mathcal{S}^{2n+1}_+ \\
    & \Leftrightarrow &
    \left[
        \begin{array}{ccc}
            A^{-2}  & w-b        & Z \\
            (w-b)^T & 1-\lambda  & . \\
            Z^T     & .          & \lambda I_n
        \end{array}
    \right]\in\mathcal{S}^{2n+1}_+
\end{eqnarray}
Since $A$ is semidefinite, we can everywhere replace $A$ by $A^{-1}$, so 
\begin{eqnarray}
C &=& \{ x\in\mathbb{R}^n : ||A^{-1}(x-b)||^2 \leq 1 \} \\
  &=& \{ Au+b: ||u||^2\leq 1 \} 
\end{eqnarray}
and 
\begin{eqnarray}
E \subset C &\Leftrightarrow& 
    \left[
        \begin{array}{ccc}
            A^2  & w-b           & Z \\
            (w-b)^T & 1-\lambda  & . \\
            Z^T     & .          & \lambda I_n
        \end{array}
    \right]\in\mathcal{S}^{2n+1}_+
\end{eqnarray}



\subsection{Alternate form (Boyd and Vandenberghe)}
If we instead define the ellipsoid as 
\[
    C(P,q) = \{ x\in\mathbb{R}^n : ||Px+q|| \leq 1 \}
\]
and the contained ellipsoid as 
\[
    \mathcal{E} (A,b,c) = \{ x : x^TAx + 2b^Tb + c \leq 0\}
\]
we get
\begin{eqnarray}
    && \mathcal{E} (Z,w) \subset C(A,b) \\
    &\Leftrightarrow& 
    \left[
        \begin{array}{ccc}
            \lambda A -P^2 & \lambda b-Pq& 0 \\
            (\lambda b-Pq)^T & (1+\lambda c) & -(Ab)^T \\
            0 & -Ab & A^2
        \end{array}
    \right]\in\mathcal{S}^{2n+1}_+
\end{eqnarray}

\subsection{Bound: Ellipsoid containing point}
For an ellipsoid 
\[
    C(A,b) = \{ x\in\mathbb{R}^n : ||A(x-b)|| \leq 1 \}
\]
A constraint for this to contain a point $p$, is quite simple:
\begin{eqnarray}
    || A(p-b) ||^2 \leq 1 & \Leftrightarrow & \left[ \begin{array}{c} 1 \\ A(p-b) \end{array}\right]\in\ \mathcal{Q}^{n+1}
\end{eqnarray}



\subsection{Bound: Ellipsoid contained in ellipsoid}
For an ellipsoid 
\[
    E(A,b) = \{ Zx+w |\ ||x||\leq 1 \}
\]
the constraint $E(A,b) \subset \{ x: || B(x-c) || \leq 1\}$ can be modeled as 
\begin{eqnarray}
    \left[
        \begin{array}{ccc}
            I_n & B(x-c) & BZ \\
            (B(x-c))^T & 1-\lambda & 0 \\
            (BZ)^T & 0 & \lambda I_n
        \end{array}
    \right] &\in & \mathcal{S}^{2n+1}_+\\
    Z & \in & \mathcal{S}^n_+\\
    \lambda & \geq & 0
\end{eqnarray}

Note that because of the ellipsoid representation, maximizing $\det(Z)$ will
maximize the volume.

\section{$\mathcal{S}$-lemma}

For $A,B\in\mathcal{S}^n$, 
\begin{eqnarray}
    x^tAx\geq 0 & \Leftrightarrow & x^tBx\geq 0
\end{eqnarray}
if and only if
\begin{eqnarray}
    &&\exists \lambda\geq 0: B-\lambda A \in \mathcal{S}^n_+ \\
    &\Leftrightarrow & \exists \lambda\geq 0, \forall x : x^T Bx-\lambda x^TAx \geq 0
\end{eqnarray}

See Nemirovski and Ben-Tal for proof.

Also, from Boyd and Vandenberghe, The S-procedure:

For symmetric $A_1,A_2\in\mathcal{S}^n$, $b_1,b_2\in\mathbb{R}^n$, $c_1,c_2\in\mathbb{R}$, if
\begin{eqnarray}
    \exists \hat{x}: \hat{x}^TA_2\hat{x} + 2b_2^T\hat{x} + c_2 < 0
\end{eqnarray}
then 
\begin{eqnarray}
    x^T A_1x + 2b_1^Tx + c_1 < 0 &,&  x^T A_2x + 2b_2^Tx + c_2 < 0 \\
    &\Updownarrow& \\
    \not\exists \lambda\geq 0:
\end{eqnarray}




\section{Lemma on Schur complement}
\begin{eqnarray}
    A & = & \left[\begin{array}{cc} B & C^T \\ C & D\end{array}\right],\ B\in\mathcal{S}^k_+,\ D\in\mathbb{R}^{l\times l}
\end{eqnarray}
Then 
\begin{eqnarray}
    A\in\mathcal{S}^{k+l} & \Leftrightarrow & D-CB^{-1}C^T \mathcal{S}^{l}
\end{eqnarray}

See Nemirovski and Ben-Tal for proof.

\end{document}
