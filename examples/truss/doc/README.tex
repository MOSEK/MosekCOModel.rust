\documentclass{article}

\RequirePackage{mathtools}
\RequirePackage{amsfonts}

\begin{document}

\section{Truss design problem definition}

The linear truss problem as defined by Nemirovski and Ben-Tal,
Lectures on Modern Convex Optimization, Problem 3.4.1). A truss consists of a
set of nodes (with geometric positions), a subset of which are fixed, a set of
bars (arcs) that connect
the nodes and an external load applied to a subset of nodes. The problem is
then to assign sizes to arcs to minimize "compliancy", basically stress on the
structure.

We have the following parameters for the problem
\begin{itemize}
\item $\mathrm{d}$ is the number if dimensions (2 or 3).
\item $N$ set of nodes, with $p_i$ being the position of node $i\in N$, with
    $N_f\subset N$ being the set of fixed nodes, and $N_u=N\backslash N_f$ the
        set of free nodes.
\item $A\subseteq N\times N$. We are using the convention that if $(i,j)\in A$
    then $(j,i)\not\in A$ (i.e only one bar per two nodes). 
\item $w$ is the total available volume available of material for bars.
\item $\kappa\in\mathbb{R}$ material constant defining elasticity
\item $F$ is set of forces, $f\in F: f\in\mathbb{R}^{|N|\times\mathrm{d}}$ is the vector of external forces on each node.
\item $b_{ij}\in\mathbb{R}^{|N|\times\mathrm{d}}$ for $(i,j)\in A$ is a value that is computed from $\kappa$, $p_i$ and $p_j$. $b_{ij}$ is defined as 
    \[
        b_{(ij),k} = \left\{ 
            \begin{array}{l}
                \beta_{ij}\ \mathrm{ if }\ j = k\hat j\not\in N_f \\
                -\beta_{ij}\ \mathrm{ if }\ i = k\hat j\not\in N_f \\
                0\  \mathrm{ otherwise}
            \end{array}
        \right.
    \]
\item $\beta_{ij}\in\mathbb{R}^{|N|\times\mathrm{d}}$ defined assign
    \[
        \beta_{ij} = \sqrt\kappa \frac{p_j-p_i}{||p_j-p_i||^2}
    \]
\end{itemize}

Then we define the problem as 

\begin{eqnarray}
    \mathrm{minimize} &&  \tau \\
    \mathrm{such\ that} \\
    && \left[
        \begin{array}{l}
            (t_a,\sigma_a,s_a)\in\mathcal{Q}^3_r,\ a\in A\\
            \displaystyle \tau \geq \sum_{a\in A} \sigma_a \\
            \displaystyle \sum_{a\in A} t_a \leq w \\
            \displaystyle \sum_{a\in A} s_a b_a = f \\
            t\in\mathbb{R}^{|A|}\\
            \sigma\in\mathbb{R}^{|A|} \\
            s\in\mathbb{R}^{|A|} 
        \end{array}
    \right]_{f\in F} \\
    && \tau\in\mathbb{R}\\
\end{eqnarray}
The block is repeated for each force vector in $F$.

Note that the constraint on $f$ can amounts to for each node $i\in N_u$
\begin{eqnarray}
    f_i & = & \sum_{a\in A} s_{a} b_{a,i} \\
        & = & \sum_{j\in N_u:(i,j)\in A} s_{ij} \beta_{ij} 
            - \sum_{j\in N_u:(j,i)\in A} s_{ij} \beta_{ij}
\end{eqnarray}
Basically, the force applied at node $i$ must be equal to the weighted sum of
stresses on all bars connected to node $i$. Note entirely unlike a flow
conservation constraint in a non-directed graph.


\subsection{Computing displacements}
The compliance for force $f$ and material volume $t$ is defined as 
\[
    \mathrm{Compl}_f(t) = \max_t \left(f^Tv  - \frac{1}{2} v^TAv\right)
\]
For an optimal $t$, we have 
\begin{eqnarray}
    \nabla (v^TAv-f^Tv) = 0 & \Rightarrow & Av=f
\end{eqnarray}
If $A$ is positive definite, we can obtain $v$ from $f$.

Here, the matrix $A=\sum_{a\in A} B_a^TB_a$, where $B_a$ is the $|A| \times
d|N|$ matrix we built earlier from $b_{ij,k}$ (each $beta_{ij,k}$ is a
$d$-vector, so to get a single vector for $b_{ij}$ we flatten this into
$b_{ij}\in\mathbb{R}^{d|A|}$). In other words each
$B_a\in\mathbb{R}^{d|N|\times d|N|}$ with $(2D)^2$ (symmetric) entries. For $a=(i,j)$ and $D=2$ we get 
\[
    B_{ij} = t_{ij}\kappa\left[
        \begin{array}{ccccccc}
            \ddots  & 0_{*ix}           & 0_{*iy}           & \cdots & 0_{*jx}           & 0_{*jy}           & \cdots \\
            0_{i*x} & b_{ij,x}^2        & b_{ij,x}b_{ij,y}  & \cdots & -b_{ij,x}^2       & -b_{ij,x}b_{ij,y} & \cdots \\
            0_{i*y} & b_{ij,x}b_{ij,y}  & b_{ij,y}^2        & \cdots & -b_{ij,x}b_{ij,y} & - b_{ij,y}^2      & \cdots \\
            \vdots  & \vdots            & \vdots            & \ddots & \vdots            & \vdots            & \cdots \\
            0_{j*x} & -b_{ij,x}^2       & -b_{ij,x}b_{ij,y} & .      & b_{ij,x}^2        & b_{ij,x}b_{ij,y}  & \cdots \\
            0_{j*y} & -b_{ij,x}b_{ij,y} & -b_{ij,y}^2       & .      & b_{ij,x}b_{ij,y}  & b_{ij,y}^2        & \cdots \\
            \vdots  & \vdots            & \vdots            & \vdots & \vdots            & \vdots            & \ddots 
        \end{array}
    \right]
\]
If $|A| \geq D|N|$ (and all arcs are linearly independant), this should be positive definite, and we can obtain the 

\end{document}

