\documentclass{article}

\RequirePackage{mathtools}
\RequirePackage{amsfonts}

\begin{document}

\section{Truss design problem definition}

The linear truss problem as defined by Nemirovski and Ben-Tal,
Lectures on Modern Convex Optimization, Problem 3.4.1). A truss consists of a
set of nodes (with geometric positions), a subset of which are fixed, a set of bars (arcs) that connect
the nodes and an external load applied to a subset of nodes. The problem is
then to assign sizes to arcs to minimize "compliancy", basically stress on the
structure.

We have the following parameters for the problem
\begin{itemize}
\item $N$ set of nodes, with $p_i$ being the position of node $i\in N$, with $N_f\subset N$ being the set of fixed nodes.
\item $A\subseteq N\times N$. We are using the convention that if $(i,j)\in A$
    then $(j,i)\not\in A$ (i.e only one bar per two nodes). 
\item $w$ is the total available volume available of material for bars.
\item $\kappa$ matreial constant
\item $b_{ij}\in\mathbb{R}^{|N|}$ for $(i,j)\in A$ is a value that is computed from $\kappa$, $p_i$ and $p_j$. $b_{ij}$ is defined as 
    \[
        b_{ijk} = \left\{ 
            \begin{array}{l}
                \beta_{ij} \mathrm{ if } j = k\hat j\not\in N_f \\
                -\beta_{ij} \mathrm{ if } i = k\hat j\not\in N_f \\
                0 \mathrm{ otherwise}
            \end{array}
        \right.
    \]
\end{itemize}
Then we define the problem as 
\begin{eqnarray}
    \mathrm{minimize} &&  \tau \\
    \mathrm{such that} 
    && (t_a,\sigma_a,s_a)\in\mathcal{Q}^3_r,\ a\in A\\
    && \tau \geq \sum_{a\in A} \sigma_a \\
    && \sum_{a\in A} t_a \leq w \\
    && \sum_{a\in A} s_a b_a = f \\
    && t\in\mathbb{R}^{|A|}\\
    && \sigma\in\mathbb{R}^{|A|} \\
    && \tau\in\mathbb{R}\\
    && s\in\mathbb{R}^{|A|} 
\end{eqnarray}


\end{document}

